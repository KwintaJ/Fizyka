\documentclass[14pt, table]{extarticle}
\usepackage{amsfonts}
\usepackage{amsmath}
\usepackage[utf8]{inputenc}
\usepackage[a4paper, total={7in, 10.5in}]{geometry}
\usepackage[table]{xcolor}
\usepackage{tgbonum}
\usepackage{tikz}
\usepackage{circuitikz}
\usepackage[T1]{fontenc}
\usetikzlibrary{quotes,angles}
\usetikzlibrary{arrows}

\newcommand{\metr}{\textrm{m}}
\newcommand{\sekunda}{\textrm{s}}
\newcommand{\gram}{\textrm{g}}
\newcommand{\farrad}{\textrm{F}}
\newcommand{\wolt}{\textrm{V}}
\newcommand{\kulomb}{\textrm{C}}
\newcommand{\ohm}{\Omega}
\newcommand{\tesla}{\textrm{T}}
\newcommand{\newton}{\textrm{N}}


\title{Fizyka \\ \Large{Zestaw 8}}
\author{Jan Kwinta}
\date{2023-01-03}

\begin{document}
\maketitle

\paragraph{Zadanie 1.}
Jaka siła Lorentza działa na proton, który z predkością $\vec{v} = (v_0, 0, 0)$ wpada w pole magnetyczne o indukcji $\vec{B} = (0, B_0, 0)$ ? Ładunek protonu wynosi $e = 1.6 \cdot 10^{-19} \kulomb, B_0 = 2 \tesla$ i $v_0 = 108 \frac{\metr}{\sekunda}$.

$$ \vec{F} = q \cdot \left( \vec{v} \times \vec{B} \right) $$ \\
$$ \vec{F} = q \cdot \left( 
\begin{bmatrix}
	v_0 \\
	0 \\
	0 \\
\end{bmatrix}
\times 
\begin{bmatrix}
	0 \\
	B_0 \\
	0 \\
\end{bmatrix} 
\right) = q \cdot \det 
\begin{bmatrix}
	\hat{i} &\ \hat{j} & \hat{k} \\
	v_0 & 0 & 0 \\
	0 & B_0 & 0 \\
\end{bmatrix} $$ \\
$$ \vec{F} = q \cdot 
\begin{bmatrix}
	0 \\
	0 \\
	v_0 \cdot B_0 \\
\end{bmatrix} = 
\begin{bmatrix}
	0 \\
	0 \\
	q \cdot v_0 \cdot B_0\\
\end{bmatrix} =
\begin{bmatrix}
	0 \\
	0 \\
	F_0 \\
\end{bmatrix} $$ \\
$$ \textrm{gdzie } F_0 = q \cdot v_0 \cdot B_0 = 3.456 \cdot 10^{-17} \newton $$



\newpage
\paragraph{Zadanie 2.}
Udowodnić, że energia kinetyczna naładowanej czastki poruszającej się w polu magnetycznym jest stała w czasie.


\newpage
\paragraph{Zadanie 3.}
Udowodnić, że całkowita siła działajaca na zamknięty obwód z prądem w jednorodnym polu magnetycznym wynosi zero. Obwód ma dowolny kształt i nie musi zawierać się w jednej płaszczyźnie. 


\newpage
\paragraph{Zadanie 5.}
W nieskończenie długim walcu o promieniu $R$ płynie prad o stałej gęstści $J$. Korzystajac z prawa Ampère’a znaleźć indukcję magnetyczną $\vec{B}$ w odległości $r$ od osi walca w dwóch przypadkach: $r \leq R$ i $r > R$\\ \\
$$ J = \frac{I}{S} \Rightarrow I = JS = J \pi R^2 $$ \\ 
(a) $r \leq R$ \\ \\
Indukcja wewnątrz przewodnika: przez kontur kołowy o promieniu $r$ przepływa jakiś prąd $I_r$ będący częścią prądu $I$. \\
$$ I_r = I \frac{\pi r^2}{\pi R^2} = J \pi R^2 \frac{\pi r^2}{\pi R^2} = J \pi r^2 $$ \\
Z prawa Ampère’a:
$$ \oint B dl = \mu_0 I \Rightarrow B2 \pi r = \mu_0 I$$
$$ B = \frac{\mu_0 I_r}{2 \pi r} = \frac{J \pi r^2}{2 \pi r} = \frac{1}{2} J r \mu_0 $$ \\ \\
(b) $r > R$ \\ \\
Indukcja na zewnątrz przewodnika: możemy uprościć model - sprowadzamy walec o promieniu $R$ do cienkiego przewodu. Wtedy, z prawa Ampère’a:
$$ B = \frac{\mu_0 I}{2 \pi r} = \frac{J \pi R^2}{2 \pi r} = \frac{1}{2} \cdot \frac{J R^2}{r}$$ 


\newpage
\paragraph{Zadanie 7.}
Kwadratową ramkę o boku $a$ i całkowitym oporze $R$ umieszczono w odległości $s$ od
nieskończonego przewodnika liniowego, w którym płynie prąd $I(t)$

\begin{equation*}
I(t) = \left\{
        \begin{array}{ll}
            (1 - \alpha t) I_0 \quad &\text{,} \, \ 1 \leq t \leq \frac{1}{\alpha} \\
            \\
            0 \quad &\text{,} \, \ t > \frac{1}{\alpha}
        \end{array}
    \right.
\end{equation*} \\
gdzie $\alpha$ i $I_0$ to dodatnie stałe. Ramka i przewodnik leżą w jednej płaszczyźnie, a bok ramki jest równoległy do przewodnika. Jaka bedzie wartość natężenia i kierunek prądu indukowanego w ramce prądu $I_i(t)$ ?



\newpage
\paragraph{Zadanie 8.}
Dany jest tzw. szeregowy obwód RLC. Znaleźć równanie różniczkowe opisujace napiecie na kondensatorze $U(t)$ i jego zwiazek z natężeniem prądu $I$ płynacego w obwodzie. W ogólnym przypadku w obwód można wpiać źródło zewnętrznej siły elektromotorycznej zmiennej w czasie $\epsilon(t)$. Co stanowi mechaniczny odpowiednik
takiego obwodu? Dlaczego zwykle rozważania ograniczaja się do siły elektromotorycznej postaci $\epsilon(t) = \epsilon_0 \cos(\omega t)$ lub $\epsilon(t) = \epsilon_0 \sin(\omega t)$, gdzie $\epsilon_0$ i $\omega$ to stałe?

\begin{center}
\begin{circuitikz}

\draw (0, 0)
	  to [cute inductor] (3, 0)
	  to [R=$R$] (6, 0)
	  to [C=$C$] (9, 0)
	  to (9, -1.5)
	  to (4, -1.5);

\draw (0, -1.5) [short,-o] to (2.75, -1.5);
	  
\draw (0, -1.5) to (0, 0);

\draw (4, -1.5) [short,-o] to (3.25, -1.5);

\draw
	  (1.5, 0.3) node [anchor=south] {$L$}
	  (3, -1.5) node [anchor=north] {$\epsilon$};

\end{circuitikz}
\end{center}

Obecność oporu w obwodzie powoduje straty energii i zawarta w obwodzie energia maleje i wzrasta - otrzymujemy drgania tłumione analogiczne do drgań oscylatora harmonicznego tłumionego. Współczynnik tłumienia równy jest $\frac{R}{2L}$.\\ \\
Zasilanie takiego obwodu zmienną siłą elektromotoryczną (ze stałym okresem) prowadzi do podtrzymania drgań. \\

$$ U(t) = \frac{1}{C} \cdot \int I \ dt $$ \\ \\
Z prawa Kirchhoffa:
$$ \frac{dI}{dt}L + I\cdot R + \frac{Q}{C} = \epsilon(t) $$
Wiedząc, że $I = \frac{dQ}{dt}$ możemy podstawić i zróżniczkować obustronnie:
$$ \frac{d^2I}{dt^2} L + \frac{dI}{dt} R + \frac{I}{C} = \frac{dI}{dt} \epsilon(t) $$\\ \\
Powiedzmy, że  $\epsilon(t) = \epsilon_0 \sin(\omega t)$. Wtedy:
$$ \frac{dI}{dt} \epsilon(t) = \omega \epsilon_0 \cos(\omega t) $$ \\
Więc:
$$ \frac{d^2I}{dt^2} + \frac{dI}{dt} \frac{R}{L} + \frac{I}{LC} = \frac{\omega \epsilon_0}{L} \cos(\omega t) $$



\end{document}