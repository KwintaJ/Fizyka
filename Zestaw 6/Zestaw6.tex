\documentclass[14pt, table]{extarticle}
\usepackage{amsfonts}
\usepackage{amsmath}
\usepackage[utf8]{inputenc}
\usepackage[a4paper, total={7in, 10.5in}]{geometry}
\usepackage[table]{xcolor}
\usepackage{tgbonum}
\usepackage{tikz}
\usepackage{circuitikz}
\usepackage[T1]{fontenc}
\usetikzlibrary{quotes,angles}
\usetikzlibrary{arrows}

\newcommand{\farrad}{\textrm{F}}
\newcommand{\wolt}{\textrm{V}}
\newcommand{\kulomb}{\textrm{C}}

\title{Fizyka \\ \Large{Zestaw 6}}
\author{Jan Kwinta}
\date{2022-12-20}

\begin{document}
\maketitle

\paragraph{Zadanie 2.}
Obliczyć potencjał $V(x, y, z)$ jednorodnie naładowanego (ze stałą gęstością powierzchniową ładunku $\sigma$) cienkiego dysku o środku w poczatku układu i promieniu $R$ leżącego w płaszczyźnie $xy$. \\ \\
(a) znaleźć ogólne wyrażenie (przy pomocy całki) na potencjał w dowolnym punkcie
$P(x, y, z)$ \\ \\
$$ V_{P}(x, y, z) = k \int \frac{\sigma d S}{\left| \vec{OP} - \vec{\xi} \  \right|} $$
$$ V_{P}(x, y, z) = k \int_0^R \frac{\sigma 2 \pi r \ d r}{\sqrt{\left(x - \xi_1 \right)^2 + \left(y - \xi_2 \right)^2 + \left(z - \xi_3 \right)^2 }} $$
$$ V_{P}(x, y, z) = 2k \pi \sigma \int_0^R \frac{r \ d r}{\sqrt{\left(x - \xi_1 \right)^2 + \left(y - \xi_2 \right)^2 + \left(z - \xi_3 \right)^2 }} $$ \\ \\
Dla dowolnego punktu $P(0, 0, z)$ całka z powyższego podpunktu ma postać:
$$ V_{P}(0, 0, z) = k \int_0^R \frac{\sigma 2 \pi r}{\sqrt{z^2 + r^2}} \ d r$$
$$ V_{P}(0, 0, z) = 2 k \pi \sigma \int_0^R \frac{r}{\sqrt{z^2 + r^2}} \ d r = 2 k \pi \sigma \left( \sqrt{R^2 + z^2} - \sqrt{z^2} \right)$$
\\ \\

\newpage
\paragraph{Zadanie 5.}
Policzyć energię elektrostatyczną układu trzech ładunków punktowych $Q_1 = 1e$,
$Q_2 = 4e$, i $Q_3 = 2e$, umieszczonych odpowiednio w punktach $P_1 = (0, 0, 0)$, $P_2 =
(0, 4, 0) 10^{-10}$m i $P_3 = (3, 0, 0) 10^{-10}$m.
\\ \\
$$ E_{123} = \frac{k}{2} \left[ Q_1\left( \frac{kQ_2}{r_{12}} + \frac{kQ_3}{r_{13}} \right) + Q_2\left( \frac{kQ_1}{r_{12}} + \frac{kQ_3}{r_{23}} \right) + Q_3\left( \frac{kQ_1}{r_{13}} + \frac{kQ_2}{r{23}} \right) \right] $$
\\ \\
Obliczmy:
$$ r_{12} = \left| \vec{r_1} - \vec{r_2} \right| = \sqrt{\left( 0 - 0 \right)^2 + \left( 0 - 4 \cdot 10^{-10} \right)^2 + \left( 0 - 0 \right)^2} = 4 \cdot 10^{-10} \textrm{m} $$ \\
$$ r_{13} = \left| \vec{r_1} - \vec{r_3} \right| = \sqrt{\left( 0 - 3 \cdot 10^{-10} \right)^2 + \left( 0 - 0 \right)^2 + \left( 0 - 0 \right)^2} = 3 \cdot 10^{-10} \textrm{m} $$ \\
$$ r_{23} = \left| \vec{r_2} - \vec{r_3} \right| = \sqrt{\left( 0 - 3 \cdot 10^{-10} \right)^2 + \left(4 \cdot 10^{-10} - 0 \right)^2 + \left( 0 - 0 \right)^2} = $$
$$ = \sqrt{10^{-20} \left( 9 + 16 \right) } = 5 \cdot 10^{-10} \textrm{m} $$
\\ \\
Zatem energia elektrostatyczna układu jest równa:
$$ E_{123} = \frac{k}{2} \Biggl[ 1e \left( \frac{k\cdot 4e }{ 4 \cdot 10^{-10} } + \frac{k\cdot 2e }{ 3 \cdot 10^{-10} } \right) + $$
$$ + 4e \left( \frac{k \cdot 1e }{ 4 \cdot 10^{-10} } + \frac{k\cdot 2e }{ 5 \cdot 10^{-10} } \right) + 2e \left( \frac{k\cdot 1e }{ 3 \cdot 10^{-10} } + \frac{k\cdot 4e}{ 5 \cdot 10^{-10} } \right) \Biggr] $$ \\ 
$$ E_{123} = \frac{k}{2} \Biggl[ 1e \left( \frac{3 k\cdot 4e + 4 k\cdot 2e}{ 12 \cdot 10^{-10} } \right) + $$
$$ + 4e \left( \frac{5k \cdot 1e + 4k\cdot 2e }{ 20 \cdot 10^{-10} } \right) + 2e \left( \frac{5k\cdot 1e + 3k\cdot 4e}{ 15 \cdot 10^{-10} } \right) \Biggr] $$ \\ 
\newpage
$$ E_{123} = \frac{k}{2} \left[ 1e \left( \frac{12ke + 8ke}{ 12 \cdot 10^{-10} } \right) + 4e \left( \frac{5ke + 8ke}{ 20 \cdot 10^{-10} } \right) + 2e \left( \frac{5ke + 12ke}{ 15 \cdot 10^{-10} } \right) \right] $$ \\ 
$$ E_{123} = \frac{k}{2} \left( \frac{20ke^2}{ 12 \cdot 10^{-10} } + \frac{52ke^2}{ 20 \cdot 10^{-10} }  + \frac{34ke^2}{ 15 \cdot 10^{-10} } \right) $$ \\ 
$$ E_{123} = k^2e^2 \left( \frac{10}{12} + \frac{26}{20}  + \frac{17}{15} \right) \frac{1}{10^{-10}} $$ \\ 
$$ E_{123} = \frac{49}{15 \cdot 10^{-10}} k^2e^2 = 3.2(6) \cdot 10^{10} \cdot k^2e^2 $$




\newpage
\paragraph{Zadanie 6.}
Układ trzech kondensatorów o pojemnościach $C_1 = 1 \mu$F, $C_2 = 2 \mu$F i $C_3 = 3 \mu$F został podłaczony do stałego napięcia $U = 1$V. \\ Znaleźć ładunek, napiecie i energię
elektrostatyczną dla każdego z kondensatorów oraz pojemność zastępczą układu.


\begin{center}
\begin{circuitikz}

\draw (0, -3)
	  to (0, 0)
      to (3, 0)
      to (3, 1)
      to [C=$C_1$] (5, 1)
      to (5, 0)
      to (7, 0);

\draw (3, 0)
	  to (3, -1)
	  to [C=$C_2$] (5, -1)
	  to (5, 0);

\draw (7, 0)
	  to [C=$C_3$] (9, 0)
	  to (10, 0)
	  to (10, -3);

\draw (4.75, -3)
	  to [short,o-] (0, -3);


\draw (5.25, -3)
	  to [short,o-] (10, -3);

\draw
	  (5, -3) node [anchor=north] {$U$};

\end{circuitikz}
\end{center}

$U_{1+2}$ - napięcie na połączonych szeregowo kondensatorach $1$ i $2$. \\

$U_1$ - napięcie na kondensatorze $1$. \\

$U_2$ - napięcie na kondensatorze $2$. \\

$U_3$ - napięcie na kondensatorze $3$. \\

$C_{Z12}$ - pojemność zastępcza za kondensatory $1$ i $2$. \\ 

$C_{Z123}$ - pojemność zastępcza za wszystkie kondensatory. \\

$Q_C$ - ładunek całkowity. \\

$Q_{1+2}$ - ładunek na połączonych szeregowo kondensatorach $1$ i $2$. \\

$Q_1$ - ładunek kondensatora $1$. \\

$Q_2$ - ładunek kondensatora $2$. \\

$Q_3$ - ładunek kondensatora $3$. \\

\newpage
$$ U = U_{1+2} + U_3 = 1 \wolt $$ 
$$ C_{Z12} = C_1 + C_2 = 3 \mu\farrad $$ 
$$ \frac{1}{C_{Z123}} = \frac{1}{C_3} + \frac{1}{C_{Z12}} $$ \\ 
$$ C_{Z123} = \frac{C_3C_{Z12}}{C_3 + C_{Z12}} = \frac{3 \cdot 3}{3 + 3} = \frac{9}{6} = 1,5 \mu\farrad $$ \\ 
Łączenie szeregowe:
$$ Q_C = C_{Z123} \cdot U = 1,5 \mu\kulomb $$
$$ Q_{1+2} = Q_3 = Q_C = 1,5 \mu\kulomb $$ \\
$$ C_3 = \frac{Q_3}{U_3} \Rightarrow U_3 = \frac{Q_3}{C_3} = \frac{1,5 \mu\kulomb}{3 \mu\farrad} = 0,5 \wolt $$ \\
$$ U_{1+2} = \frac{Q_3}{C_3} = U - U_3 = 0,5 \wolt $$ \\
Łączenie równoległe:
$$ Q_{1+2} = Q_1 + Q_2 $$
$$ U_1 = U_2 = U_{1+2} = 0,5 \wolt$$ \\
$$ Q_1 = U_1 \cdot C_1 = 0,5 \wolt \cdot 1 \mu\farrad = 0,5 \mu\kulomb $$
$$ Q_2 = U_2 \cdot C_2 = 0,5 \wolt \cdot 2 \mu\farrad = 1 \mu\kulomb $$ \\
Obliczmy energię każdego z kondensatorów:
$$ E_1 = \frac{C_1 \cdot U_1^2}{2} = \frac{1 \mu\farrad \cdot \left(0,5 \wolt \right)^2}{2} = 0,125 \mu \textrm{J} $$ 
$$ E_2 = \frac{C_2 \cdot U_2^2}{2} = \frac{2 \mu\farrad \cdot \left(0,5 \wolt \right)^2}{2} = 0,25 \mu \textrm{J} $$ 
$$ E_3 = \frac{C_3 \cdot U_3^2}{2} = \frac{3 \mu\farrad \cdot \left(0,5 \wolt \right)^2}{2} = 0,375 \mu \textrm{J} $$ 



\end{document}